% Options for packages loaded elsewhere
\PassOptionsToPackage{unicode}{hyperref}
\PassOptionsToPackage{hyphens}{url}
%
\documentclass[
]{book}
\usepackage{lmodern}
\usepackage{amssymb,amsmath}
\usepackage{ifxetex,ifluatex}
\ifnum 0\ifxetex 1\fi\ifluatex 1\fi=0 % if pdftex
  \usepackage[T1]{fontenc}
  \usepackage[utf8]{inputenc}
  \usepackage{textcomp} % provide euro and other symbols
\else % if luatex or xetex
  \usepackage{unicode-math}
  \defaultfontfeatures{Scale=MatchLowercase}
  \defaultfontfeatures[\rmfamily]{Ligatures=TeX,Scale=1}
\fi
% Use upquote if available, for straight quotes in verbatim environments
\IfFileExists{upquote.sty}{\usepackage{upquote}}{}
\IfFileExists{microtype.sty}{% use microtype if available
  \usepackage[]{microtype}
  \UseMicrotypeSet[protrusion]{basicmath} % disable protrusion for tt fonts
}{}
\makeatletter
\@ifundefined{KOMAClassName}{% if non-KOMA class
  \IfFileExists{parskip.sty}{%
    \usepackage{parskip}
  }{% else
    \setlength{\parindent}{0pt}
    \setlength{\parskip}{6pt plus 2pt minus 1pt}}
}{% if KOMA class
  \KOMAoptions{parskip=half}}
\makeatother
\usepackage{xcolor}
\IfFileExists{xurl.sty}{\usepackage{xurl}}{} % add URL line breaks if available
\IfFileExists{bookmark.sty}{\usepackage{bookmark}}{\usepackage{hyperref}}
\hypersetup{
  pdftitle={Code style guide},
  pdfauthor={Yang Cao},
  hidelinks,
  pdfcreator={LaTeX via pandoc}}
\urlstyle{same} % disable monospaced font for URLs
\usepackage{color}
\usepackage{fancyvrb}
\newcommand{\VerbBar}{|}
\newcommand{\VERB}{\Verb[commandchars=\\\{\}]}
\DefineVerbatimEnvironment{Highlighting}{Verbatim}{commandchars=\\\{\}}
% Add ',fontsize=\small' for more characters per line
\usepackage{framed}
\definecolor{shadecolor}{RGB}{248,248,248}
\newenvironment{Shaded}{\begin{snugshade}}{\end{snugshade}}
\newcommand{\AlertTok}[1]{\textcolor[rgb]{0.94,0.16,0.16}{#1}}
\newcommand{\AnnotationTok}[1]{\textcolor[rgb]{0.56,0.35,0.01}{\textbf{\textit{#1}}}}
\newcommand{\AttributeTok}[1]{\textcolor[rgb]{0.77,0.63,0.00}{#1}}
\newcommand{\BaseNTok}[1]{\textcolor[rgb]{0.00,0.00,0.81}{#1}}
\newcommand{\BuiltInTok}[1]{#1}
\newcommand{\CharTok}[1]{\textcolor[rgb]{0.31,0.60,0.02}{#1}}
\newcommand{\CommentTok}[1]{\textcolor[rgb]{0.56,0.35,0.01}{\textit{#1}}}
\newcommand{\CommentVarTok}[1]{\textcolor[rgb]{0.56,0.35,0.01}{\textbf{\textit{#1}}}}
\newcommand{\ConstantTok}[1]{\textcolor[rgb]{0.00,0.00,0.00}{#1}}
\newcommand{\ControlFlowTok}[1]{\textcolor[rgb]{0.13,0.29,0.53}{\textbf{#1}}}
\newcommand{\DataTypeTok}[1]{\textcolor[rgb]{0.13,0.29,0.53}{#1}}
\newcommand{\DecValTok}[1]{\textcolor[rgb]{0.00,0.00,0.81}{#1}}
\newcommand{\DocumentationTok}[1]{\textcolor[rgb]{0.56,0.35,0.01}{\textbf{\textit{#1}}}}
\newcommand{\ErrorTok}[1]{\textcolor[rgb]{0.64,0.00,0.00}{\textbf{#1}}}
\newcommand{\ExtensionTok}[1]{#1}
\newcommand{\FloatTok}[1]{\textcolor[rgb]{0.00,0.00,0.81}{#1}}
\newcommand{\FunctionTok}[1]{\textcolor[rgb]{0.00,0.00,0.00}{#1}}
\newcommand{\ImportTok}[1]{#1}
\newcommand{\InformationTok}[1]{\textcolor[rgb]{0.56,0.35,0.01}{\textbf{\textit{#1}}}}
\newcommand{\KeywordTok}[1]{\textcolor[rgb]{0.13,0.29,0.53}{\textbf{#1}}}
\newcommand{\NormalTok}[1]{#1}
\newcommand{\OperatorTok}[1]{\textcolor[rgb]{0.81,0.36,0.00}{\textbf{#1}}}
\newcommand{\OtherTok}[1]{\textcolor[rgb]{0.56,0.35,0.01}{#1}}
\newcommand{\PreprocessorTok}[1]{\textcolor[rgb]{0.56,0.35,0.01}{\textit{#1}}}
\newcommand{\RegionMarkerTok}[1]{#1}
\newcommand{\SpecialCharTok}[1]{\textcolor[rgb]{0.00,0.00,0.00}{#1}}
\newcommand{\SpecialStringTok}[1]{\textcolor[rgb]{0.31,0.60,0.02}{#1}}
\newcommand{\StringTok}[1]{\textcolor[rgb]{0.31,0.60,0.02}{#1}}
\newcommand{\VariableTok}[1]{\textcolor[rgb]{0.00,0.00,0.00}{#1}}
\newcommand{\VerbatimStringTok}[1]{\textcolor[rgb]{0.31,0.60,0.02}{#1}}
\newcommand{\WarningTok}[1]{\textcolor[rgb]{0.56,0.35,0.01}{\textbf{\textit{#1}}}}
\usepackage{longtable,booktabs}
% Correct order of tables after \paragraph or \subparagraph
\usepackage{etoolbox}
\makeatletter
\patchcmd\longtable{\par}{\if@noskipsec\mbox{}\fi\par}{}{}
\makeatother
% Allow footnotes in longtable head/foot
\IfFileExists{footnotehyper.sty}{\usepackage{footnotehyper}}{\usepackage{footnote}}
\makesavenoteenv{longtable}
\usepackage{graphicx,grffile}
\makeatletter
\def\maxwidth{\ifdim\Gin@nat@width>\linewidth\linewidth\else\Gin@nat@width\fi}
\def\maxheight{\ifdim\Gin@nat@height>\textheight\textheight\else\Gin@nat@height\fi}
\makeatother
% Scale images if necessary, so that they will not overflow the page
% margins by default, and it is still possible to overwrite the defaults
% using explicit options in \includegraphics[width, height, ...]{}
\setkeys{Gin}{width=\maxwidth,height=\maxheight,keepaspectratio}
% Set default figure placement to htbp
\makeatletter
\def\fps@figure{htbp}
\makeatother
\setlength{\emergencystretch}{3em} % prevent overfull lines
\providecommand{\tightlist}{%
  \setlength{\itemsep}{0pt}\setlength{\parskip}{0pt}}
\setcounter{secnumdepth}{5}
\usepackage{booktabs}
\usepackage[]{natbib}
\bibliographystyle{plainnat}

\title{Code style guide}
\author{Yang Cao}
\date{2021-12-25}

\begin{document}
\maketitle

{
\setcounter{tocdepth}{1}
\tableofcontents
}
\hypertarget{about}{%
\chapter*{About}\label{about}}
\addcontentsline{toc}{chapter}{About}

Welcome!

Write code in a consistent style will make it easier to maintain, and easier
for collaborators to understand. Code is more often read than written, and code
style can also have a great impact on the readability of your code

The style guides here are fundamentally opinionated, but not too opinionated.
And many decisions are arbitrary, any suggestions are welcome.

The bash is derived from
\href{https://google.github.io/styleguide/shellguide.html}{Google's shell style guide};
python code style is derived from
\href{https://www.python.org/dev/peps/pep-0008/}{Google's python style guide} and
\href{https://www.python.org/dev/peps/pep-0008/}{PEP 8 style guide for python}; and
the R code style is drived from
\href{https://style.tidyverse.org/}{Tidyverse Style Guide}.

\textbf{If you're modifying existing files, you should follow the style that's already
in the files.}

\begin{verbatim}
Do what Romans do in Rome.
\end{verbatim}

\hypertarget{bash-shell-style-guide}{%
\chapter{Bash Shell Style Guide}\label{bash-shell-style-guide}}

\hypertarget{comments}{%
\section{Comments}\label{comments}}

\hypertarget{file-header}{%
\subsection{File Header}\label{file-header}}

Start each file with a description of its contents.

Every file must have a top-level comment including a brief overview of its
contents. Author information are optional.

Example:

\begin{Shaded}
\begin{Highlighting}[]
\CommentTok{#!/bin/bash}
\CommentTok{# }
\CommentTok{# Author: }
\CommentTok{# Desc: Perform hot backups of Oracle databases.}
\end{Highlighting}
\end{Shaded}

\hypertarget{function-comments}{%
\subsection{Function Comments}\label{function-comments}}

Any function that is not both obvious and short must be commented. Any function
in a library must be commented regardless of length or complexity.

It should be possible for someone else to learn how to use your program or to
use a function in your library by reading the comments (and self-help, if
provided) without reading the code.

All function comments should contain:

\begin{itemize}
\tightlist
\item
  Description of the function
\item
  Global variables used and modified
\item
  Arguments taken
\item
  Returned values other than the default exit status of the last command run
\end{itemize}

Example:

\begin{Shaded}
\begin{Highlighting}[]
\CommentTok{#!/bin/bash}
\CommentTok{# }
\CommentTok{# Author:}
\CommentTok{# Desc: Cleanup files from the backup directory.}
\CommentTok{#######################################}
\CommentTok{# Globals:}
\CommentTok{#   BACKUP_DIR}
\CommentTok{#   ORACLE_SID}
\CommentTok{# Arguments:}
\CommentTok{#   None}
\CommentTok{#######################################}
\KeywordTok{function}\FunctionTok{ cleanup()} \KeywordTok{\{}
\NormalTok{  …}
\KeywordTok{\}}
\end{Highlighting}
\end{Shaded}

\hypertarget{implementation-comments}{%
\subsection{Implementation Comments}\label{implementation-comments}}

Don't comment everything. Comment tricky, non-obvious, interesting or important
parts of your code. use comments to explain the ``why'' not the ``what'' or ``how''.
Each line of a comment should begin with the comment symbol and a single space:
\texttt{\#}.

\hypertarget{naming-conventions}{%
\section{Naming conventions}\label{naming-conventions}}

\hypertarget{file-names}{%
\subsection{File Names}\label{file-names}}

Lowercase, with underscores to separate words if desired. `

\hypertarget{function-names}{%
\subsection{Function Names}\label{function-names}}

Function names should be in snake\_case. That is, all lower case and words are
separated by underscores. Parentheses are required after the function name.

\begin{Shaded}
\begin{Highlighting}[]
\FunctionTok{func()} \KeywordTok{\{}
  \ExtensionTok{...}
\KeywordTok{\}}

\CommentTok{### Variable Names}

\ExtensionTok{As}\NormalTok{ for function names. Variables names for loops should be similarly named for }
\ExtensionTok{any}\NormalTok{ variable you’re looping through.}

\KeywordTok{```}\FunctionTok{sh}
\KeywordTok{for} \ExtensionTok{zone}\NormalTok{ in }\StringTok{"}\VariableTok{$\{zones[@]\}}\StringTok{"}\KeywordTok{;} \KeywordTok{do}
  \ExtensionTok{something_with} \StringTok{"}\VariableTok{$\{zone\}}\StringTok{"}
\KeywordTok{done}
\end{Highlighting}
\end{Shaded}

\hypertarget{constants-and-environment-variable-names}{%
\subsection{Constants and Environment Variable Names}\label{constants-and-environment-variable-names}}

All caps, separated with underscores, declared at the top of the file.

\hypertarget{formatting}{%
\section{Formatting}\label{formatting}}

\hypertarget{identation}{%
\subsection{Identation}\label{identation}}

Indent 2 spaces. No tabs.

Use blank lines between blocks to improve readability. Indentation is two
spaces. Whatever you do, don't use tabs. For existing files, stay faithful to
the existing indentation.

\hypertarget{line-length}{%
\subsection{Line Length}\label{line-length}}

Maximum line length is 80 characters.

\hypertarget{pipe}{%
\subsection{Pipe}\label{pipe}}

Pipelines should be split one per line if they don't all fit on one line, and
put pipe symbol (\texttt{\textbar{}}) at the beginning of of its statement

If a pipeline all fits on one line, it should be on one line.

\begin{Shaded}
\begin{Highlighting}[]
 \CommentTok{# This is an inline pipe: "$(ls -la /foo/ | grep /bar/)"}

 \CommentTok{# The following pipe is of display form: every command is on}
 \CommentTok{# its own line}

\VariableTok{foobar=}\StringTok{"}\VariableTok{$(} \KeywordTok{\textbackslash{}}
  \FunctionTok{ls}\NormalTok{ -la /foo/ \textbackslash{}}
  \KeywordTok{|} \FunctionTok{grep}\NormalTok{ /bar/ \textbackslash{}}
  \KeywordTok{|} \FunctionTok{awk} \StringTok{'\{print $NF\}'}\VariableTok{)}\StringTok{"}
\end{Highlighting}
\end{Shaded}

\hypertarget{loops}{%
\subsection{Loops}\label{loops}}

Put \texttt{;\ do} and \texttt{;\ then} on the same line as the \texttt{while}, \texttt{for} or \texttt{if}. \texttt{else}
should be on its own line and closing statements should be on their own line
vertically aligned with the opening statement.

\hypertarget{variable-expansion}{%
\subsection{Variable Expansion}\label{variable-expansion}}

In order of precedence: Stay consistent with what you find; quote your variables;
prefer \texttt{"\$\{var\}"} over \texttt{"\$var"}.

They are listed in order of precedence.

\begin{itemize}
\item
  Stay consistent with what you find for existing code.
\item
  Quote variables, see Quoting section below.
\item
  Don't brace-delimit single character shell specials / positional parameters,
  unless strictly necessary or avoiding deep confusion.

\begin{Shaded}
\begin{Highlighting}[]
\CommentTok{# Section of *recommended* cases.}

\CommentTok{# Preferred style for 'special' variables:}
\BuiltInTok{echo} \StringTok{"Positional: }\VariableTok{$1}\StringTok{"} \StringTok{"}\VariableTok{$5}\StringTok{"} \StringTok{"}\VariableTok{$3}\StringTok{"}
\BuiltInTok{echo} \StringTok{"Specials: !=}\VariableTok{$!}\StringTok{, -=}\VariableTok{$-}\StringTok{, _=}\VariableTok{$_}\StringTok{. ?=}\VariableTok{$?}\StringTok{, #=}\VariableTok{$#}\StringTok{ *=}\VariableTok{$*}\StringTok{ @=}\VariableTok{$@}\StringTok{ }\DataTypeTok{\textbackslash{}$}\StringTok{=}\VariableTok{$$}\StringTok{ …"}

\CommentTok{# Braces necessary:}
\BuiltInTok{echo} \StringTok{"many parameters: $\{10\}"}

\CommentTok{# Braces avoiding confusion:}
\CommentTok{# Output is "a0b0c0"}
\KeywordTok{set} \ExtensionTok{--}\NormalTok{ a b c}
\BuiltInTok{echo} \StringTok{"}\VariableTok{$\{1\}}\StringTok{0}\VariableTok{$\{2\}}\StringTok{0}\VariableTok{$\{3\}}\StringTok{0"}
\end{Highlighting}
\end{Shaded}
\end{itemize}

Prefer brace-delimiting all other variables.

\hypertarget{quoting}{%
\section{Quoting}\label{quoting}}

\begin{itemize}
\item
  Always quote strings containing variables, command substitutions, spaces or
  shell meta characters.
\item
  Optionally quote shell-internal, readonly special variables that are defined
  to be integers: \texttt{\$?}, \texttt{\$\#}, \texttt{\$\$}, \texttt{\$!}.
\item
  Use double quotes for strings that require variable expansion or command
  substitution interpolation, and single quotes for all others.

\begin{Shaded}
\begin{Highlighting}[]
\CommentTok{# "Double" quotes indicate that substitution is required/tolerated.}

\CommentTok{# "quote variables"}
\BuiltInTok{echo} \StringTok{"}\VariableTok{$\{flag\}}\StringTok{"}

\CommentTok{# double quote for strings that require variable expansion }
\VariableTok{bar=}\StringTok{"You are }\VariableTok{$USER}\StringTok{"}
\CommentTok{# or command substitution}
\VariableTok{number=}\StringTok{"}\VariableTok{$(}\ExtensionTok{generate_number}\VariableTok{)}\StringTok{"}

\CommentTok{# single quote for strings does not require variable expansion}
\VariableTok{foo=}\StringTok{'Hello World'}

\CommentTok{# "quote shell meta characters"}
\BuiltInTok{echo} \StringTok{'Hello stranger, and well met. Earn lots of $$$'}
\BuiltInTok{echo} \StringTok{"Process }\VariableTok{$$}\StringTok{: Done making }\DataTypeTok{\textbackslash{}$\textbackslash{}$\textbackslash{}$}\StringTok{."}
\end{Highlighting}
\end{Shaded}
\end{itemize}

\hypertarget{error-handing}{%
\section{Error Handing}\label{error-handing}}

All errors should be sent to \texttt{STDERR}.

\hypertarget{error-checking}{%
\subsection{Error Checking}\label{error-checking}}

\texttt{cd}, for example, doesn't always work. Make sure to check for any possible
errors for \texttt{cd} (or commands like it) and exit or break if they are present.

\begin{Shaded}
\begin{Highlighting}[]
\CommentTok{# wrong}
\BuiltInTok{cd}\NormalTok{ /some/path }\CommentTok{# this could fail}
\FunctionTok{rm}\NormalTok{ file       }\CommentTok{# if cd fails where am I? what am I deleting?}

\CommentTok{# right}
\BuiltInTok{cd}\NormalTok{ /some/path }\KeywordTok{||} \BuiltInTok{exit}
\end{Highlighting}
\end{Shaded}

\hypertarget{set--e}{%
\subsection{set -e}\label{set--e}}

Use \texttt{set\ -e} if your script is being used for your own business. Recommend do
not use it.

\begin{Shaded}
\begin{Highlighting}[]
\CommentTok{# If _do_some_critical_check fails, the script just exits and the following }
\CommentTok{# code is just skipped without any notice.}
\KeywordTok{set} \ExtensionTok{-e}
\ExtensionTok{_do_some_critical_check}

\KeywordTok{if [[} \VariableTok{$?} \OtherTok{-ge}\NormalTok{ 1}\KeywordTok{ ]]}\NormalTok{; }\KeywordTok{then}
  \BuiltInTok{echo} \StringTok{"Oh, you will never see this line."}
\KeywordTok{fi}
\end{Highlighting}
\end{Shaded}

\hypertarget{set--u}{%
\subsection{set -u}\label{set--u}}

To make sure you won't use any undeclared variable , \texttt{set\ -u} is recommended.

\hypertarget{features}{%
\section{Features}\label{features}}

\hypertarget{command-substitution}{%
\subsection{Command Substitution}\label{command-substitution}}

Use \texttt{\$(command)} instead of backticks.

\begin{Shaded}
\begin{Highlighting}[]
\CommentTok{# This is preferred}
\VariableTok{foo=}\KeywordTok{`}\FunctionTok{date}\KeywordTok{`} 

\CommentTok{# This is not}
\VariableTok{foo=$(}\FunctionTok{date}\VariableTok{)} 
\end{Highlighting}
\end{Shaded}

\hypertarget{math}{%
\subsection{Math}\label{math}}

Always use \texttt{((\ \ldots{}\ ))} or \texttt{\$((\ \ldots{}\ ))} rather than \texttt{let} or \texttt{\${[}\ \ldots{}\ {]}} or \texttt{expr}.

\begin{Shaded}
\begin{Highlighting}[]
\CommentTok{# Simple calculation used as text - note the use of $(( … )) within}
\CommentTok{# a string.}
\BuiltInTok{echo} \StringTok{"}\VariableTok{$((}\NormalTok{ 2 + 2 }\VariableTok{))}\StringTok{ is 4"}

\CommentTok{# When performing arithmetic comparisons for testing}
\KeywordTok{if} \KeywordTok{((}\NormalTok{ a < b }\KeywordTok{))}\NormalTok{; }\KeywordTok{then}
\NormalTok{  …}
\KeywordTok{fi}

\CommentTok{# Some calculation assigned to a variable.}
\KeywordTok{((}\NormalTok{ i = 10 * j + 400 }\KeywordTok{))}
\end{Highlighting}
\end{Shaded}

\hypertarget{listing-files}{%
\subsection{Listing Files}\label{listing-files}}

Listing Files
Do not parse \texttt{ls(1)}, instead use bash builtin functions to loop files

\begin{Shaded}
\begin{Highlighting}[]
\CommentTok{# use}
\KeywordTok{for} \ExtensionTok{f}\NormalTok{ in *}\KeywordTok{;} \KeywordTok{do}
    \ExtensionTok{...}
\KeywordTok{done}

\CommentTok{# not}
\KeywordTok{for} \ExtensionTok{f}\NormalTok{ in }\VariableTok{$(}\FunctionTok{ls}\VariableTok{)}\KeywordTok{;} \KeywordTok{do}
    \ExtensionTok{...}
\KeywordTok{done}
\end{Highlighting}
\end{Shaded}

\hypertarget{arrays-and-lists}{%
\subsection{Arrays and lists}\label{arrays-and-lists}}

Use bash arrays instead of a string separated by spaces (or newlines, tabs,
etc.) whenever possible.

\begin{Shaded}
\begin{Highlighting}[]
\CommentTok{# use array}
\VariableTok{modules=(}\NormalTok{json httpserver jshint}\VariableTok{)}
\KeywordTok{for} \ExtensionTok{module}\NormalTok{ in }\StringTok{"}\VariableTok{$\{modules[@]\}}\StringTok{"}\KeywordTok{;} \KeywordTok{do}
    \ExtensionTok{npm}\NormalTok{ install -g }\StringTok{"}\VariableTok{$module}\StringTok{"}
\KeywordTok{done}

\CommentTok{# instead of string separated by spaces }
\VariableTok{modules=}\StringTok{'json httpserver jshint'}
\KeywordTok{for} \ExtensionTok{module}\NormalTok{ in }\VariableTok{$modules}\KeywordTok{;} \KeywordTok{do}
    \ExtensionTok{npm}\NormalTok{ install -g }\StringTok{"}\VariableTok{$module}\StringTok{"}
\KeywordTok{done}
\end{Highlighting}
\end{Shaded}

\hypertarget{test-...-and-...}{%
\subsection{\texorpdfstring{Test, \texttt{{[}...{]}}, and \texttt{{[}{[}...{]}{]}}}{Test, {[}...{]}, and {[}{[}...{]}{]}}}\label{test-...-and-...}}

\texttt{{[}{[}\ \ldots{}\ {]}{]}} is preferred over \texttt{{[}\ \ldots{}\ {]}}, \texttt{test}. \texttt{{[}{[}\ \ldots{}\ {]}{]}} reduces errors as no
pathname expansion or word splitting takes place between \texttt{{[}{[}} and \texttt{{]}{]}}.
In addition, \texttt{{[}{[}\ \ldots{}\ {]}{]}} allows for regular expression matching, while \texttt{{[}\ \ldots{}\ {]}}
does not.

\begin{Shaded}
\begin{Highlighting}[]
\CommentTok{# This ensures the string on the left is made up of characters in}
\CommentTok{# the alnum character class followed by the string name.}
\CommentTok{# Note that the RHS should not be quoted here.}
\KeywordTok{if [[} \StringTok{"filename"}\NormalTok{ =~ ^[[:alnum:]]+name}\KeywordTok{ ]]}\NormalTok{; }\KeywordTok{then}
  \BuiltInTok{echo} \StringTok{"Match"}
\KeywordTok{fi}

\CommentTok{# This matches the exact pattern "f*" (Does not match in this case)}
\KeywordTok{if [[} \StringTok{"filename"} \OtherTok{==} \StringTok{"f*"}\KeywordTok{ ]]}\NormalTok{; }\KeywordTok{then}
  \BuiltInTok{echo} \StringTok{"Match"}
\KeywordTok{fi}
\CommentTok{# This gives a "too many arguments" error as f* is expanded to the}
\CommentTok{# contents of the current directory}
\KeywordTok{if}\BuiltInTok{ [} \StringTok{"filename"} \OtherTok{==}\NormalTok{ f*}\BuiltInTok{ ]}\NormalTok{; }\KeywordTok{then}
  \BuiltInTok{echo} \StringTok{"Match"}
\KeywordTok{fi}
\end{Highlighting}
\end{Shaded}

\hypertarget{references}{%
\section{References}\label{references}}

\begin{itemize}
\tightlist
\item
  \href{https://google.github.io/styleguide/shellguide.html}{Google shell style guide}
\item
  \href{https://github.com/bahamas10/bash-style-guide}{bash style guide}
\item
  \href{https://github.com/icy/bash-coding-style}{bash coding style}
\end{itemize}

\hypertarget{resources}{%
\section{Resources}\label{resources}}

\href{https://github.com/koalaman/shellcheck}{Shellcheck}

\hypertarget{python-style-guide}{%
\chapter{Python Style Guide}\label{python-style-guide}}

\hypertarget{comments-and-docstrings}{%
\section{Comments and Docstrings}\label{comments-and-docstrings}}

You should use comments to document code as it's written. It is important to
document your code so that you, and any collaborators, can understand it. When
you or someone else reads a comment, they should be able to easily understand
the code the comment applies to and how it fits in with the rest of your code.

\hypertarget{block-comments}{%
\subsection{Block Comments}\label{block-comments}}

Block comments generally apply to some (or all) code that follows them, and are
indented to the same level as that code. Each line of a block comment starts
with a \texttt{\#} and a single space (unless it is indented text inside the comment).

Paragraphs inside a block comment are separated by a line containing a single \#.

\hypertarget{inline-comments}{%
\subsection{Inline Comments}\label{inline-comments}}

Use inline comments sparingly.

An inline comment is a comment on the same line as a statement. Inline comments
should be separated by at least two spaces from the statement. They should start
with a \texttt{\#} and a single space.

Inline comments are unnecessary and in fact distracting if they state the
obvious. Don't do this:

\begin{Shaded}
\begin{Highlighting}[]
\NormalTok{x }\OperatorTok{=}\NormalTok{ x }\OperatorTok{+} \DecValTok{1}                 \CommentTok{# Increment x}
\end{Highlighting}
\end{Shaded}

But sometimes, this is useful:

\begin{Shaded}
\begin{Highlighting}[]
\NormalTok{x }\OperatorTok{=}\NormalTok{ x }\OperatorTok{+} \DecValTok{1}                 \CommentTok{# Compensate for border}
\end{Highlighting}
\end{Shaded}

\hypertarget{docstrings}{%
\subsection{Docstrings}\label{docstrings}}

\begin{itemize}
\item
  Write docstrings for all public modules, functions, classes, and methods.
\item
  Docstrings are not necessary for non-public methods, but you should have a
  comment that describes what the method does. This comment should appear after
  the def line.
\item
  Surround docstrings with three double quotes on either side, as in ""``This is
  a docstring''"".
\item
  Put the """ that ends a multiline docstring on a line by itself:

\begin{Shaded}
\begin{Highlighting}[]
\CommentTok{"""Return a foobang}

\CommentTok{Optional plotz says to frobnicate the bizbaz first.}
\CommentTok{"""}
\end{Highlighting}
\end{Shaded}
\item
  For one liner docstrings, please keep the closing """ on the same line:

\begin{Shaded}
\begin{Highlighting}[]
\CommentTok{"""Return an ex-parrot."""}
\end{Highlighting}
\end{Shaded}
\item
  Recommend to use
  \href{https://numpydoc.readthedocs.io/en/latest/format.html}{Numpy Docstring format}.
  An example from numpy:

\begin{Shaded}
\begin{Highlighting}[]
\CommentTok{"""Docstring for the example.py module.}

\CommentTok{Modules names should have short, all-lowercase names.  The module name may}
\CommentTok{have underscores if this improves readability.}

\CommentTok{Every module should have a docstring at the very top of the file.  The}
\CommentTok{module's docstring may extend over multiple lines.  If your docstring does}
\CommentTok{extend over multiple lines, the closing three quotation marks must be on}
\CommentTok{a line by itself, preferably preceded by a blank line.}

\CommentTok{"""}
\ImportTok{from}\NormalTok{ __future__ }\ImportTok{import}\NormalTok{ division, absolute_import, print_function}

\ImportTok{import}\NormalTok{ os  }\CommentTok{# standard library imports first}

\CommentTok{# Do NOT import using *, e.g. from numpy import *}
\CommentTok{#}
\CommentTok{# Import the module using}
\CommentTok{#}
\CommentTok{#   import numpy}
\CommentTok{#}
\CommentTok{# instead or import individual functions as needed, e.g}
\CommentTok{#}
\CommentTok{#  from numpy import array, zeros}
\CommentTok{#}
\CommentTok{# If you prefer the use of abbreviated module names, we suggest the}
\CommentTok{# convention used by NumPy itself::}

\ImportTok{import}\NormalTok{ numpy }\ImportTok{as}\NormalTok{ np}
\ImportTok{import}\NormalTok{ matplotlib }\ImportTok{as}\NormalTok{ mpl}
\ImportTok{import}\NormalTok{ matplotlib.pyplot }\ImportTok{as}\NormalTok{ plt}

\CommentTok{# These abbreviated names are not to be used in docstrings; users must}
\CommentTok{# be able to paste and execute docstrings after importing only the}
\CommentTok{# numpy module itself, unabbreviated.}


\KeywordTok{def}\NormalTok{ foo(var1, var2, }\OperatorTok{*}\NormalTok{args, long_var_name}\OperatorTok{=}\StringTok{'hi'}\NormalTok{, only_seldom_used_keyword}\OperatorTok{=}\DecValTok{0}\NormalTok{, }\OperatorTok{**}\NormalTok{kwargs):}
    \CommentTok{r"""Summarize the function in one line.}

\CommentTok{    Several sentences providing an extended description. Refer to}
\CommentTok{    variables using back-ticks, e.g. `var`.}

\CommentTok{    Parameters}
\CommentTok{    ----------}
\CommentTok{    var1 : array_like}
\CommentTok{        Array_like means all those objects -- lists, nested lists, etc. --}
\CommentTok{        that can be converted to an array.  We can also refer to}
\CommentTok{        variables like `var1`.}
\CommentTok{    var2 : int}
\CommentTok{        The type above can either refer to an actual Python type}
\CommentTok{        (e.g. ``int``), or describe the type of the variable in more}
\CommentTok{        detail, e.g. ``(N,) ndarray`` or ``array_like``.}
\CommentTok{    *args : iterable}
\CommentTok{        Other arguments.}
\CommentTok{    long_var_name : \{'hi', 'ho'\}, optional}
\CommentTok{        Choices in brackets, default first when optional.}

\CommentTok{    Returns}
\CommentTok{    -------}
\CommentTok{    type}
\CommentTok{        Explanation of anonymous return value of type ``type``.}
\CommentTok{    describe : type}
\CommentTok{        Explanation of return value named `describe`.}
\CommentTok{    out : type}
\CommentTok{        Explanation of `out`.}
\CommentTok{    type_without_description}

\CommentTok{    Other Parameters}
\CommentTok{    ----------------}
\CommentTok{    only_seldom_used_keyword : int, optional}
\CommentTok{        Infrequently used parameters can be described under this optional}
\CommentTok{        section to prevent cluttering the Parameters section.}
\CommentTok{    **kwargs : dict}
\CommentTok{        Other infrequently used keyword arguments. Note that all keyword}
\CommentTok{        arguments appearing after the first parameter specified under the}
\CommentTok{        Other Parameters section, should also be described under this}
\CommentTok{        section.}

\CommentTok{    Raises}
\CommentTok{    ------}
\CommentTok{    BadException}
\CommentTok{        Because you shouldn't have done that.}

\CommentTok{    See Also}
\CommentTok{    --------}
\CommentTok{    numpy.array : Relationship (optional).}
\CommentTok{    numpy.ndarray : Relationship (optional), which could be fairly long, in}
\CommentTok{                    which case the line wraps here.}
\CommentTok{    numpy.dot, numpy.linalg.norm, numpy.eye}

\CommentTok{    Notes}
\CommentTok{    -----}
\CommentTok{    Notes about the implementation algorithm (if needed).}

\CommentTok{    This can have multiple paragraphs.}

\CommentTok{    You may include some math:}

\CommentTok{    .. math:: X(e^\{j\textbackslash{}omega \} ) = x(n)e^\{ - j\textbackslash{}omega n\}}

\CommentTok{    And even use a Greek symbol like :math:`\textbackslash{}omega` inline.}

\CommentTok{    References}
\CommentTok{    ----------}
\CommentTok{    Cite the relevant literature, e.g. [1]_.  You may also cite these}
\CommentTok{    references in the notes section above.}

\CommentTok{    .. [1] O. McNoleg, "The integration of GIS, remote sensing,}
\CommentTok{       expert systems and adaptive co-kriging for environmental habitat}
\CommentTok{       modelling of the Highland Haggis using object-oriented, fuzzy-logic}
\CommentTok{       and neural-network techniques," Computers & Geosciences, vol. 22,}
\CommentTok{       pp. 585-588, 1996.}

\CommentTok{    Examples}
\CommentTok{    --------}
\CommentTok{    These are written in doctest format, and should illustrate how to}
\CommentTok{    use the function.}

\CommentTok{    >>> a = [1, 2, 3]}
\CommentTok{    >>> print([x + 3 for x in a])}
\CommentTok{    [4, 5, 6]}
\CommentTok{    >>> print("a}\CharTok{\textbackslash{}n}\CommentTok{b")}
\CommentTok{    a}
\CommentTok{    b}
\CommentTok{    """}
    \CommentTok{# After closing class docstring, there should be one blank line to}
    \CommentTok{# separate following codes (according to PEP257).}
    \CommentTok{# But for function, method and module, there should be no blank lines}
    \CommentTok{# after closing the docstring.}
    \ControlFlowTok{pass}
\end{Highlighting}
\end{Shaded}
\end{itemize}

\hypertarget{naming-conventions-1}{%
\section{Naming Conventions}\label{naming-conventions-1}}

\texttt{module\_name}, \texttt{package\_name}, \texttt{ClassName}, \texttt{method\_name}, \texttt{ExceptionName},
\texttt{function\_name}, \texttt{GLOBAL\_CONSTANT\_NAME}, \texttt{global\_var\_name}, \texttt{instance\_var\_name},
\texttt{function\_parameter\_name}, \texttt{local\_var\_name}.

Function names, variable names, and filenames should be descriptive. In
particular, do not use abbreviations that are ambiguous or unfamiliar to
readers outside your project, and do not abbreviate by deleting letters within
a word.

\hypertarget{names-to-avoid}{%
\subsection{Names to Avoid}\label{names-to-avoid}}

\begin{itemize}
\tightlist
\item
  single character names, except for specifically allowed cases:

  \begin{itemize}
  \tightlist
  \item
    counters or iterators (e.g.~\texttt{i}, \texttt{j}, \texttt{k}, \texttt{v}, et al.).
  \item
    \texttt{e} as an exception identifier in try/except statements.
  \item
    \texttt{f} as a file handle in with statements.
  \end{itemize}
\item
  dashes (\texttt{-}) in any package/module name.
\item
  \texttt{\_\_double\_leading\_and\_trailing\_underscore\_\_} names (reserved by Python).
\end{itemize}

\hypertarget{file-naming}{%
\subsection{File Naming}\label{file-naming}}

Python filenames must have a \texttt{.py} extension and must not contain dashes (\texttt{-}).

\hypertarget{naming-guides-from-google-python-style}{%
\subsubsection{\texorpdfstring{\href{https://google.github.io/styleguide/pyguide.html\#s3.16.4-guidelines-derived-from-guidos-recommendations}{Naming guides from google python style}}{Naming guides from google python style}}\label{naming-guides-from-google-python-style}}

\begin{longtable}[]{@{}lll@{}}
\toprule
Type & Public & Internal\tabularnewline
\midrule
\endhead
Packages & \texttt{lower\_with\_under} &\tabularnewline
Classes & \texttt{CapWords} & \texttt{\_CapWords}\tabularnewline
Exceptions & \texttt{CapWords} &\tabularnewline
Functions & \texttt{lower\_with\_under()} & \texttt{lower\_with\_under()}\tabularnewline
Global/Class Constants & \texttt{CAPS\_WITH\_UNDER} & \texttt{\_CAPS\_WITH\_UNDER}\tabularnewline
Global/Class Variables & \texttt{lower\_with\_under} & \texttt{\_lower\_with\_under}\tabularnewline
Instance Variables & \texttt{lower\_with\_under} & \texttt{\_lower\_with\_under}\tabularnewline
Method Names & \texttt{lower\_with\_under()} & \texttt{\_lower\_with\_under()}\tabularnewline
Function/Method Parameters & \texttt{lower\_with\_under} &\tabularnewline
Local Variables & \texttt{lower\_with\_under} &\tabularnewline
\bottomrule
\end{longtable}

\hypertarget{code-layout}{%
\section{Code Layout}\label{code-layout}}

\hypertarget{line-length-1}{%
\subsection{Line length}\label{line-length-1}}

\begin{itemize}
\item
  Maximum line length is 80 characters.
\item
  Explicit exceptions to the 80 character limit:

  \begin{itemize}
  \tightlist
  \item
    Long import statements.
  \item
    URLs, pathnames, or long flags in comments.
  \end{itemize}

\begin{Shaded}
\begin{Highlighting}[]
\CommentTok{# long URL on their own line if necessary. }
\CommentTok{# http:/www.example.com/longurl/xxxxxxxxxxxxxxxxxxxxxxxxxxxxxxxxxxxxxxxxxxxxxxxxxxxxxxxxxxxxxxxxxxxxx}
\end{Highlighting}
\end{Shaded}
\item
  When a literal string won't fit on a single line, use parentheses for implicit
  line joining.

\begin{Shaded}
\begin{Highlighting}[]
\NormalTok{x }\OperatorTok{=}\NormalTok{ (}\StringTok{'This will build a very long long '}
     \StringTok{'long long long long long long string'}\NormalTok{)}
\end{Highlighting}
\end{Shaded}
\item
  Python will assume line continuation if code is contained within parentheses,
  brackets, or braces:

\begin{Shaded}
\begin{Highlighting}[]
\KeywordTok{def}\NormalTok{ function(arg_one, arg_two,}
\NormalTok{             arg_three, arg_four):}
    \ControlFlowTok{return}\NormalTok{ arg_one}
\end{Highlighting}
\end{Shaded}
\item
  If it is impossible to use implied continuation, then you can use backslashes
  to break lines instead. However, if you can use implied continuation, then
  you should do so.

\begin{Shaded}
\begin{Highlighting}[]
\ImportTok{from}\NormalTok{ mypkg }\ImportTok{import}\NormalTok{ example1, }\OperatorTok{\textbackslash{}}
\NormalTok{    example2, example3}
\end{Highlighting}
\end{Shaded}
\item
  If line breaking needs to occur around binary operators, like + and *, it
  should occur before the operator.

\begin{Shaded}
\begin{Highlighting}[]
\NormalTok{total }\OperatorTok{=}\NormalTok{ (first_variable}
         \OperatorTok{+}\NormalTok{ second_variable}
         \OperatorTok{-}\NormalTok{ third_variable)}
\end{Highlighting}
\end{Shaded}
\end{itemize}

\hypertarget{indentation}{%
\subsection{Indentation}\label{indentation}}

\begin{itemize}
\item
  Use 4 spaces per indentation level.
\item
  Prefer spaces over tabs.
\item
  Never use tabs or mix tabs and spaces.

\begin{Shaded}
\begin{Highlighting}[]
\CommentTok{# Aligned with opening delimiter.}
\NormalTok{foo }\OperatorTok{=}\NormalTok{ long_function_name(var_one, var_two,}
\NormalTok{                         var_three, var_four)}

\CommentTok{# Add 4 spaces (an extra level of indentation) to distinguish arguments from }
\CommentTok{# the rest.}
\KeywordTok{def}\NormalTok{ long_function_name(}
\NormalTok{        var_one, var_two, var_three,}
\NormalTok{        var_four):}
    \BuiltInTok{print}\NormalTok{(var_one)}

\CommentTok{# Hanging indents should add a level.}
\NormalTok{foo }\OperatorTok{=}\NormalTok{ long_function_name(}
\NormalTok{    var_one, var_two,}
\NormalTok{    var_three, var_four)}
\end{Highlighting}
\end{Shaded}
\end{itemize}

\hypertarget{parentheses}{%
\subsection{Parentheses}\label{parentheses}}

\begin{itemize}
\item
  Use parentheses around tuples.
\item
  Do not use them in return statements or conditional statements
  unless using parentheses for implied line continuation or to indicate a tuple.

\begin{Shaded}
\begin{Highlighting}[]
\CommentTok{# good}
\ControlFlowTok{if}\NormalTok{ foo:}
\NormalTok{    bar()}
\ControlFlowTok{while}\NormalTok{ x:}
\NormalTok{    x }\OperatorTok{=}\NormalTok{ bar()}
\ControlFlowTok{if}\NormalTok{ x }\KeywordTok{and}\NormalTok{ y:}
\NormalTok{    bar()}
\ControlFlowTok{if} \KeywordTok{not}\NormalTok{ x:}
\NormalTok{    bar()}
\CommentTok{# For a 1 item tuple the ()s are more visually obvious than the comma.}
\NormalTok{onesie }\OperatorTok{=}\NormalTok{ (foo,)}
\ControlFlowTok{return}\NormalTok{ foo}
\ControlFlowTok{return}\NormalTok{ spam, beans}
\ControlFlowTok{return}\NormalTok{ (spam, beans)}
\ControlFlowTok{for}\NormalTok{ (x, y) }\KeywordTok{in} \BuiltInTok{dict}\NormalTok{.items(): ...}

\CommentTok{# bad}
\ControlFlowTok{if}\NormalTok{ (x):}
\NormalTok{    bar()}
\ControlFlowTok{if} \KeywordTok{not}\NormalTok{(x):}
\NormalTok{    bar()}
\ControlFlowTok{return}\NormalTok{ (foo)}
\end{Highlighting}
\end{Shaded}
\end{itemize}

\hypertarget{blank-lines}{%
\subsection{Blank Lines}\label{blank-lines}}

\begin{itemize}
\item
  Two blank lines between top-level definitions, be they function or class
  definitions.

\begin{Shaded}
\begin{Highlighting}[]
\KeywordTok{class}\NormalTok{ MyFirstClass:}
    \ControlFlowTok{pass}


\KeywordTok{class}\NormalTok{ MySecondClass:}
    \ControlFlowTok{pass}


\KeywordTok{def}\NormalTok{ top_level_function():}
    \ControlFlowTok{return} \VariableTok{None}
\end{Highlighting}
\end{Shaded}
\item
  One blank line between method definitions inside classes.

\begin{Shaded}
\begin{Highlighting}[]
\KeywordTok{class}\NormalTok{ MyClass:}
    \KeywordTok{def}\NormalTok{ first_method(}\VariableTok{self}\NormalTok{):}
        \ControlFlowTok{return} \VariableTok{None}

    \KeywordTok{def}\NormalTok{ second_method(}\VariableTok{self}\NormalTok{):}
        \ControlFlowTok{return} \VariableTok{None}
\end{Highlighting}
\end{Shaded}
\item
  Use single blank lines as you judge appropriate within functions or methods.
\end{itemize}

\hypertarget{whitespace}{%
\subsection{Whitespace}\label{whitespace}}

\begin{itemize}
\item
  No whitespace inside parentheses, brackets or braces.

\begin{Shaded}
\begin{Highlighting}[]
\CommentTok{# good}
\NormalTok{spam(ham[}\DecValTok{1}\NormalTok{], \{}\StringTok{'eggs'}\NormalTok{: }\DecValTok{2}\NormalTok{\}, [])}

\CommentTok{# bad}
\NormalTok{spam( ham[ }\DecValTok{1}\NormalTok{ ], \{ eggs: }\DecValTok{2}\NormalTok{ \} )}
\end{Highlighting}
\end{Shaded}
\item
  No whitespace before a comma, semicolon, or colon. Do use whitespace after a
  comma, semicolon, or colon, except at the end of the line.

\begin{Shaded}
\begin{Highlighting}[]
\CommentTok{# good}
\ControlFlowTok{if}\NormalTok{ x }\OperatorTok{==} \DecValTok{4}\NormalTok{: }
    \BuiltInTok{print}\NormalTok{ x, y}
\NormalTok{x, y }\OperatorTok{=}\NormalTok{ y, x}

\CommentTok{# bad}
\ControlFlowTok{if}\NormalTok{ x }\OperatorTok{==} \DecValTok{4}\NormalTok{ : }
    \BuiltInTok{print}\NormalTok{ x , y }
\NormalTok{x , y }\OperatorTok{=}\NormalTok{ y , x}
\end{Highlighting}
\end{Shaded}
\item
  No whitespace before the open paren/bracket that starts an argument list,
  indexing or slicing.

\begin{Shaded}
\begin{Highlighting}[]
\CommentTok{# good}
\NormalTok{spam(}\DecValTok{1}\NormalTok{)}
\BuiltInTok{dict}\NormalTok{[}\StringTok{'key'}\NormalTok{] }\OperatorTok{=} \BuiltInTok{list}\NormalTok{[index]}

\CommentTok{# bad}
\NormalTok{spam (}\DecValTok{1}\NormalTok{)}
\BuiltInTok{dict}\NormalTok{ [}\StringTok{'key'}\NormalTok{] }\OperatorTok{=} \BuiltInTok{list}\NormalTok{ [index]}
\end{Highlighting}
\end{Shaded}
\item
  Always surround these binary operators with a single space on either side:
  assignment (=), augmented assignment (+=, -= etc.), comparisons (==, \textless, \textgreater,
  !=, \textless\textgreater, \textless=, \textgreater=, in, not in, is, is not), Booleans (and, or, not).

\begin{Shaded}
\begin{Highlighting}[]
\CommentTok{# good}
\NormalTok{i }\OperatorTok{=}\NormalTok{ i }\OperatorTok{+} \DecValTok{1}
\NormalTok{submitted }\OperatorTok{+=} \DecValTok{1}
\NormalTok{x }\OperatorTok{=}\NormalTok{ x}\OperatorTok{*}\DecValTok{2} \OperatorTok{-} \DecValTok{1}
\NormalTok{hypot2 }\OperatorTok{=}\NormalTok{ x}\OperatorTok{*}\NormalTok{x }\OperatorTok{+}\NormalTok{ y}\OperatorTok{*}\NormalTok{y}
\NormalTok{c }\OperatorTok{=}\NormalTok{ (a}\OperatorTok{+}\NormalTok{b) }\OperatorTok{*}\NormalTok{ (a}\OperatorTok{-}\NormalTok{b)}

\CommentTok{# bad}
\NormalTok{i}\OperatorTok{=}\NormalTok{i}\OperatorTok{+}\DecValTok{1}
\NormalTok{submitted }\OperatorTok{+=}\DecValTok{1}
\NormalTok{x }\OperatorTok{=}\NormalTok{ x }\OperatorTok{*} \DecValTok{2} \OperatorTok{-} \DecValTok{1}
\NormalTok{hypot2 }\OperatorTok{=}\NormalTok{ x }\OperatorTok{*}\NormalTok{ x }\OperatorTok{+}\NormalTok{ y }\OperatorTok{*}\NormalTok{ y}
\NormalTok{c }\OperatorTok{=}\NormalTok{ (a }\OperatorTok{+}\NormalTok{ b) }\OperatorTok{*}\NormalTok{ (a }\OperatorTok{-}\NormalTok{ b)}
\end{Highlighting}
\end{Shaded}
\item
  Never use spaces around = when passing keyword arguments or defining a default
  parameter value, with one exception: when a type annotation is present, do
  use spaces around the = for the default parameter value.

\begin{Shaded}
\begin{Highlighting}[]
\CommentTok{# good}
\KeywordTok{def} \BuiltInTok{complex}\NormalTok{(real, imag}\OperatorTok{=}\FloatTok{0.0}\NormalTok{): }\ControlFlowTok{return}\NormalTok{ Magic(r}\OperatorTok{=}\NormalTok{real, i}\OperatorTok{=}\NormalTok{imag)}
\KeywordTok{def} \BuiltInTok{complex}\NormalTok{(real, imag: }\BuiltInTok{float} \OperatorTok{=} \FloatTok{0.0}\NormalTok{): }\ControlFlowTok{return}\NormalTok{ Magic(r}\OperatorTok{=}\NormalTok{real, i}\OperatorTok{=}\NormalTok{imag)}

\CommentTok{# bad}
\KeywordTok{def} \BuiltInTok{complex}\NormalTok{(real, imag }\OperatorTok{=} \FloatTok{0.0}\NormalTok{): }\ControlFlowTok{return}\NormalTok{ Magic(r }\OperatorTok{=}\NormalTok{ real, i }\OperatorTok{=}\NormalTok{ imag)}
\KeywordTok{def} \BuiltInTok{complex}\NormalTok{(real, imag: }\BuiltInTok{float}\OperatorTok{=}\FloatTok{0.0}\NormalTok{): }\ControlFlowTok{return}\NormalTok{ Magic(r }\OperatorTok{=}\NormalTok{ real, i }\OperatorTok{=}\NormalTok{ imag)}
\end{Highlighting}
\end{Shaded}
\end{itemize}

\hypertarget{trailing-commas}{%
\subsection{Trailing Commas}\label{trailing-commas}}

Trailing commas in sequences of items are recommended only when the closing
container token \texttt{{]}}, \texttt{)}, or \texttt{\}} does not appear on the same line as the final
element.

\begin{Shaded}
\begin{Highlighting}[]
\CommentTok{# good}
\NormalTok{golomb3 }\OperatorTok{=}\NormalTok{ [}\DecValTok{0}\NormalTok{, }\DecValTok{1}\NormalTok{, }\DecValTok{3}\NormalTok{]}
\NormalTok{golomb4 }\OperatorTok{=}\NormalTok{ [}
    \DecValTok{0}\NormalTok{,}
    \DecValTok{1}\NormalTok{,}
    \DecValTok{4}\NormalTok{,}
    \DecValTok{6}\NormalTok{,}
\NormalTok{]}

\CommentTok{# bad}
\NormalTok{golomb4 }\OperatorTok{=}\NormalTok{ [}\DecValTok{0}\NormalTok{, }\DecValTok{1}\NormalTok{, }\DecValTok{4}\NormalTok{, }\DecValTok{6}\NormalTok{,]}
\end{Highlighting}
\end{Shaded}

\hypertarget{strings} operator, or the \texttt{format} method for formatting
  strings, even when the parameters are all strings.
\item
  Use your best judgment to decide between + and \% (or format) though.
\item
  Do not use \% or the format method for pure concatenation.

\begin{Shaded}
\begin{Highlighting}[]
\CommentTok{# good}
\NormalTok{x }\OperatorTok{=} \StringTok{'}\SpecialCharTok{%s}\StringTok{, }\SpecialCharTok\NormalTok{ (imperative, expletive)}
\NormalTok{x }\OperatorTok{=} \StringTok{'}\SpecialCharTok{\{\}}\StringTok{, }\SpecialCharTok{\{\}}\StringTok{'}\NormalTok{.}\BuiltInTok{format}\NormalTok{(first, second)}
\NormalTok{x }\OperatorTok{=} \StringTok{'name: }\SpecialCharTok{%s}\StringTok{; score: }\SpecialCharTok\NormalTok{ (name, n)}
\NormalTok{x }\OperatorTok{=} \StringTok{'name: }\SpecialCharTok{\{\}}\StringTok{; score: }\SpecialCharTok{\{\}}\StringTok{'}\NormalTok{.}\BuiltInTok{format}\NormalTok{(name, n)}
\NormalTok{x }\OperatorTok{=} \SpecialStringTok{f'name: }\SpecialCharTok{\{}\NormalTok{name}\SpecialCharTok{\}}\SpecialStringTok{; score: }\SpecialCharTok{\{n\}}\SpecialStringTok{'}  

\CommentTok{# bad}
\NormalTok{x }\OperatorTok{=} \StringTok{'}\SpecialCharTok{%s%s}\StringTok{'} \OperatorTok{%}\NormalTok{ (a, b)  }\CommentTok{# use + in this case}
\NormalTok{x }\OperatorTok{=} \StringTok{'}\SpecialCharTok{\{\}\{\}}\StringTok{'}\NormalTok{.}\BuiltInTok{format}\NormalTok{(a, b)  }\CommentTok{# use + in this case}
\NormalTok{x }\OperatorTok{=}\NormalTok{ first }\OperatorTok{+} \StringTok{', '} \OperatorTok{+}\NormalTok{ second}
\NormalTok{x }\OperatorTok{=} \StringTok{'name: '} \OperatorTok{+}\NormalTok{ name }\OperatorTok{+} \StringTok{'; score: '} \OperatorTok{+} \BuiltInTok{str}\NormalTok{(n)}
\end{Highlighting}
\end{Shaded}
\item
  Be consistent with your choice of string quote character within a file.
  Pick \texttt{\textquotesingle{}} or \texttt{"} and stick with it. It is okay to use the other quote
  character on a string to avoid the need to backslash-escape quote characters
  within the string.
\item
  Prefer \texttt{"""} for multi-line strings rather than \texttt{\textquotesingle{}\textquotesingle{}\textquotesingle{}}.
\item
  Multi-line strings do not flow with the indentation of the rest of the program.
  If you need to avoid embedding extra space in the string, use concatenated
  single-line strings.

\begin{Shaded}
\begin{Highlighting}[]
\CommentTok{# good}
\NormalTok{long_string }\OperatorTok{=} \StringTok{"""This is fine if your use case can accept}
\StringTok{    extraneous leading spaces."""}
\NormalTok{long_string }\OperatorTok{=}\NormalTok{ (}\StringTok{"And this is fine if you cannot accept}\CharTok{\textbackslash{}n}\StringTok{"} \OperatorTok{+}
               \StringTok{"extraneous leading spaces."}\NormalTok{)}

\CommentTok{# bad}
\NormalTok{long_string }\OperatorTok{=} \StringTok{"""This is pretty ugly.}
\StringTok{Don't do this.}
\StringTok{"""}
\end{Highlighting}
\end{Shaded}
\end{itemize}

\hypertarget{imports}{%
\subsection{Imports}\label{imports}}

\begin{itemize}
\item
  Imports should usually be on separate lines.

\begin{Shaded}
\begin{Highlighting}[]
\CommentTok{# good}
\ImportTok{import}\NormalTok{ os}
\ImportTok{import}\NormalTok{ sys}

\CommentTok{# bad}
\ImportTok{import}\NormalTok{ sys, os}
\end{Highlighting}
\end{Shaded}
\item
  Imports are always put at the top of the file, just after any module comments
  and docstrings and before module globals and constants. Imports should be
  grouped in the following order, and you should put a blank line between each
  group of imports.

  \begin{itemize}
  \item
    Standard library imports.
  \item
    Related third party imports.
  \item
    Local application/library specific imports.

\begin{Shaded}
\begin{Highlighting}[]
\ImportTok{import}\NormalTok{ collections}
\ImportTok{import}\NormalTok{ sys}

\ImportTok{from}\NormalTok{ absl }\ImportTok{import}\NormalTok{ app}
\ImportTok{from}\NormalTok{ absl }\ImportTok{import}\NormalTok{ flags}

\ImportTok{from}\NormalTok{ myproject.backend }\ImportTok{import}\NormalTok{ huxley}
\ImportTok{from}\NormalTok{ myproject.backend.state_machine }\ImportTok{import}\NormalTok{ main_loop}
\end{Highlighting}
\end{Shaded}
  \end{itemize}
\end{itemize}

\hypertarget{references-1}{%
\section{References}\label{references-1}}

\begin{itemize}
\tightlist
\item
  \href{https://www.python.org/dev/peps/pep-0008/}{PEP 8 style guide for python code}
\item
  \href{https://github.com/amontalenti/elements-of-python-style}{The elements of python style}
\item
  \href{https://google.github.io/styleguide/pyguide.html}{Google python style guid}
\item
  \href{https://numpydoc.readthedocs.io/en/latest/format.html}{numpy Docstring format}
\end{itemize}

\hypertarget{r-style-guide}{%
\chapter{R Style Guide}\label{r-style-guide}}

\hypertarget{files}{%
\section{Files}\label{files}}

\hypertarget{names}{%
\subsection{Names}\label{names}}

\begin{itemize}
\item
  File names should be meaningful and end in .R. Avoid using special
  characters in file names - stick with numbers, letters, \texttt{-}, and \texttt{\_}.

\begin{Shaded}
\begin{Highlighting}[]
\CommentTok{# Good}
\NormalTok{fit_models.R}
\NormalTok{utility_functions.R}

\CommentTok{# Bad}
\NormalTok{fit models.R}
\NormalTok{foo.r}
\NormalTok{stuff.r}
\end{Highlighting}
\end{Shaded}
\item
  If files should be run in a particular order, prefix them with numbers. If
  it seems likely you'll have more than 10 files, left pad with zero:

\begin{Shaded}
\begin{Highlighting}[]
\DecValTok{00}\NormalTok{_download.R}
\DecValTok{01}\NormalTok{_explore.R}
\NormalTok{...}
\DecValTok{09}\NormalTok{_model.R}
\DecValTok{10}\NormalTok{_visualize.R}
\end{Highlighting}
\end{Shaded}
\item
  Prefer file names that are all lower case, and never have names that differ
  only in their capitalization.
\end{itemize}

\hypertarget{internal-structure}{%
\subsection{Internal structure}\label{internal-structure}}

\begin{itemize}
\item
  Use commented lines of - and = to break up your file into easily readable chunks.

\begin{Shaded}
\begin{Highlighting}[]
\CommentTok{# Load data ---------------------------}

\CommentTok{# Plot data ---------------------------}
\end{Highlighting}
\end{Shaded}
\item
  If your script uses add-on packages, load them all at once at the very
  beginning of the file.
\end{itemize}

\hypertarget{naming-convention}{%
\section{Naming Convention}\label{naming-convention}}

\begin{itemize}
\item
  Variable and function names should use only lowercase letters, numbers,
  and \texttt{\_}. Use underscores (\texttt{\_}) (so called snake case) to separate words within
  a name.

\begin{Shaded}
\begin{Highlighting}[]
\CommentTok{# Good}
\NormalTok{day_one}
\NormalTok{day_}\DecValTok{1}
\end{Highlighting}
\end{Shaded}
\item
  It's better to reserve dots exclusively for the S3 object system. In S3,
  methods are given the name \texttt{function.class}.
\item
  Generally, variable names should be nouns and function names should be
  verbs. Strive for names that are concise and meaningful (this is not easy!).

\begin{Shaded}
\begin{Highlighting}[]
\CommentTok{# Good}
\NormalTok{day_one}

\CommentTok{# Bad}
\NormalTok{first_day_of_the_month}
\NormalTok{djm1}
\end{Highlighting}
\end{Shaded}
\item
  Avoid re-using names of common functions and variables. This will cause
  confusion for the readers of your code.

\begin{Shaded}
\begin{Highlighting}[]
\CommentTok{# Bad}
\NormalTok{T <-}\StringTok{ }\OtherTok{FALSE}
\NormalTok{c <-}\StringTok{ }\DecValTok{10}
\NormalTok{mean <-}\StringTok{ }\ControlFlowTok{function}\NormalTok{(x) }\KeywordTok{sum}\NormalTok{(x)}
\end{Highlighting}
\end{Shaded}
\end{itemize}

\hypertarget{spacing}{%
\section{Spacing}\label{spacing}}

\hypertarget{indentation-1}{%
\subsection{Indentation}\label{indentation-1}}

Use four spaces.

\hypertarget{commas}{%
\subsection{Commas}\label{commas}}

Always put a space after a comma, never before, just like in regular English.

\begin{Shaded}
\begin{Highlighting}[]
\CommentTok{# Good}
\NormalTok{x[, }\DecValTok{1}\NormalTok{]}

\CommentTok{# Bad}
\NormalTok{x[,}\DecValTok{1}\NormalTok{]}
\NormalTok{x[ ,}\DecValTok{1}\NormalTok{]}
\NormalTok{x[ , }\DecValTok{1}\NormalTok{]}
\end{Highlighting}
\end{Shaded}

\hypertarget{parentheses-1}{%
\subsection{Parentheses}\label{parentheses-1}}

\begin{itemize}
\item
  Do not put spaces inside or outside parentheses for regular function calls.

\begin{Shaded}
\begin{Highlighting}[]
\CommentTok{# Good}
\KeywordTok{mean}\NormalTok{(x, }\DataTypeTok{na.rm =} \OtherTok{TRUE}\NormalTok{)}

\CommentTok{# Bad}
\KeywordTok{mean}\NormalTok{ (x, }\DataTypeTok{na.rm =} \OtherTok{TRUE}\NormalTok{)}
\KeywordTok{mean}\NormalTok{( x, }\DataTypeTok{na.rm =} \OtherTok{TRUE}\NormalTok{ )}
\end{Highlighting}
\end{Shaded}
\item
  Place a space before and after \texttt{()} when used with \texttt{if}, \texttt{for}, or \texttt{while}.

\begin{Shaded}
\begin{Highlighting}[]
\CommentTok{# Good}
\ControlFlowTok{if}\NormalTok{ (debug) \{}
    \KeywordTok{show}\NormalTok{(x)}
\NormalTok{\}}

\CommentTok{# Bad}
\ControlFlowTok{if}\NormalTok{(debug)\{}
    \KeywordTok{show}\NormalTok{(x)}
\NormalTok{\}}
\end{Highlighting}
\end{Shaded}
\item
  Place a space after \texttt{()} used for function arguments:

\begin{Shaded}
\begin{Highlighting}[]
\CommentTok{# Good}
\ControlFlowTok{function}\NormalTok{(x) \{\}}

\CommentTok{# Bad}
\ControlFlowTok{function}\NormalTok{ (x) \{\}}
\ControlFlowTok{function}\NormalTok{(x)\{\}}
\end{Highlighting}
\end{Shaded}
\end{itemize}

\hypertarget{embracing}{%
\subsection{Embracing}\label{embracing}}

The embracing operator, \texttt{\{\{\ \}\}}, should always have inner spaces to help
emphasise its special behaviour:

\begin{verbatim}
# Good
max_by <- function(data, var, by) {
  data %>%
    group_by({{ by }}) %>%
    summarise(maximum = max({{ var }}, na.rm = TRUE))
}

# Bad
max_by <- function(data, var, by) {
  data %>%
    group_by({{by}}) %>%
    summarise(maximum = max({{var}}, na.rm = TRUE))
}
\end{verbatim}

\hypertarget{infix-operators}{%
\subsection{Infix operators}\label{infix-operators}}

Most infix operators (\texttt{==}, \texttt{+}, \texttt{-}, \texttt{\textless{}-}, etc.) should always be surrounded
by spaces:

\begin{verbatim}
# Good
height <- (feet * 12) + inches
mean(x, na.rm = TRUE)

# Bad
height<-feet*12+inches
mean(x, na.rm=TRUE)
\end{verbatim}

There are a few exceptions, which should never be surrounded by spaces:

\begin{itemize}
\item
  The operators with high precedence: \texttt{::}, \texttt{:::}, \texttt{\$}, \texttt{@}, \texttt{{[}}, \texttt{{[}{[}}, \texttt{\^{}},
  unary \texttt{-}, unary \texttt{+}, and \texttt{:}.
\item
  Single-sided formulas when the right-hand side is a single identifier.

\begin{Shaded}
\begin{Highlighting}[]
\CommentTok{# Good}
\OperatorTok{~}\NormalTok{foo}

\CommentTok{# Bad}
\OperatorTok{~}\StringTok{ }\NormalTok{foo}
\end{Highlighting}
\end{Shaded}

  Note that single-sided formulas with a complex right-hand side do need a space:

\begin{Shaded}
\begin{Highlighting}[]
\CommentTok{# Good}
\OperatorTok{~}\StringTok{ }\NormalTok{.x }\OperatorTok{+}\StringTok{ }\NormalTok{.y}

\CommentTok{# Bad}
\OperatorTok{~}\NormalTok{.x }\OperatorTok{+}\StringTok{ }\NormalTok{.y}
\end{Highlighting}
\end{Shaded}
\item
  When used in tidy evaluation \texttt{!!} (bang-bang) and \texttt{!!!} (bang-bang-bang)
\end{itemize}

\hypertarget{function-calls}{%
\section{Function Calls}\label{function-calls}}

\hypertarget{named-arguments}{%
\subsection{Named arguments}\label{named-arguments}}

\begin{itemize}
\item
  If you override the default value of an argument, use the full name.
\item
  You can omit the argument names for very common arguments

\begin{Shaded}
\begin{Highlighting}[]
\CommentTok{# Good}
\KeywordTok{mean}\NormalTok{(}\DecValTok{1}\OperatorTok{:}\DecValTok{10}\NormalTok{, }\DataTypeTok{na.rm =} \OtherTok{TRUE}\NormalTok{)}

\CommentTok{# Bad}
\KeywordTok{mean}\NormalTok{(}\DataTypeTok{x =} \DecValTok{1}\OperatorTok{:}\DecValTok{10}\NormalTok{, , }\OtherTok{FALSE}\NormalTok{)}
\KeywordTok{mean}\NormalTok{(, }\OtherTok{TRUE}\NormalTok{, }\DataTypeTok{x =} \KeywordTok{c}\NormalTok{(}\DecValTok{1}\OperatorTok{:}\DecValTok{10}\NormalTok{, }\OtherTok{NA}\NormalTok{))}
\end{Highlighting}
\end{Shaded}
\end{itemize}

\hypertarget{assignment}{%
\subsection{Assignment}\label{assignment}}

Avoid assignment in function calls:

\begin{Shaded}
\begin{Highlighting}[]
\CommentTok{# Good}
\NormalTok{x <-}\StringTok{ }\KeywordTok{complicated_function}\NormalTok{()}
\ControlFlowTok{if}\NormalTok{ (}\KeywordTok{nzchar}\NormalTok{(x) }\OperatorTok{<}\StringTok{ }\DecValTok{1}\NormalTok{) \{}
  \CommentTok{# do something}
\NormalTok{\}}

\CommentTok{# Bad}
\ControlFlowTok{if}\NormalTok{ (}\KeywordTok{nzchar}\NormalTok{(x <-}\StringTok{ }\KeywordTok{complicated_function}\NormalTok{()) }\OperatorTok{<}\StringTok{ }\DecValTok{1}\NormalTok{) \{}
  \CommentTok{# do something}
\NormalTok{\}}
\end{Highlighting}
\end{Shaded}

The only exception is in functions that capture side-effects:

\begin{Shaded}
\begin{Highlighting}[]
\NormalTok{output <-}\StringTok{ }\KeywordTok{capture.output}\NormalTok{(x <-}\StringTok{ }\KeywordTok{f}\NormalTok{())}
\end{Highlighting}
\end{Shaded}

\hypertarget{long-lines}{%
\subsection{Long lines}\label{long-lines}}

There are two options if the function name and definition can't fit on a single
line:

\begin{itemize}
\item
  Function-indent: place each argument on its own line, and indent to match
  the opening \texttt{(} of function:

\begin{Shaded}
\begin{Highlighting}[]
\NormalTok{long_function_name <-}\StringTok{ }\ControlFlowTok{function}\NormalTok{(}\DataTypeTok{a =} \StringTok{"a long argument"}\NormalTok{,}
                               \DataTypeTok{b =} \StringTok{"another argument"}\NormalTok{,}
                               \DataTypeTok{c =} \StringTok{"another long argument"}\NormalTok{) \{}
  \CommentTok{# As usual code is indented by two spaces.}
\NormalTok{\}}
\end{Highlighting}
\end{Shaded}
\item
  Double-indent: Place each argument of its own double indented line.

\begin{Shaded}
\begin{Highlighting}[]
\NormalTok{long_function_name <-}\StringTok{ }\ControlFlowTok{function}\NormalTok{(}
    \DataTypeTok{a =} \StringTok{"a long argument"}\NormalTok{,}
    \DataTypeTok{b =} \StringTok{"another argument"}\NormalTok{,}
    \DataTypeTok{c =} \StringTok{"another long argument"}\NormalTok{) \{}
    \CommentTok{# As usual code is indented by two spaces.}
\NormalTok{\}}
\end{Highlighting}
\end{Shaded}
\end{itemize}

In both cases the trailing \texttt{)} and leading \texttt{\{} should go on the same line as
the last argument.

Prefer function-indent style to double-indent style when it fits.

These styles are designed to clearly separate the function definition from its
body.

\begin{verbatim}
# Bad
long_function_name <- function(a = "a long argument",
    b = "another argument",
    c = "another long argument") {
    # Here it's hard to spot where the definition ends and the
    # code begins, and to see all three function arguments
}
\end{verbatim}

\hypertarget{return}{%
\subsection{Return}\label{return}}

\begin{itemize}
\item
  Only use \texttt{return()} for early returns. Otherwise, rely on R to return the
  result of the last evaluated expression.

\begin{Shaded}
\begin{Highlighting}[]
\CommentTok{# Good}
\NormalTok{find_abs <-}\StringTok{ }\ControlFlowTok{function}\NormalTok{(x) \{}
    \ControlFlowTok{if}\NormalTok{ (x }\OperatorTok{>}\StringTok{ }\DecValTok{0}\NormalTok{) \{}
        \KeywordTok{return}\NormalTok{(x)}
\NormalTok{    \}}
\NormalTok{    x }\OperatorTok{*}\StringTok{ }\DecValTok{-1}
\NormalTok{\}}
\NormalTok{add_two <-}\StringTok{ }\ControlFlowTok{function}\NormalTok{(x, y) \{}
\NormalTok{    x }\OperatorTok{+}\StringTok{ }\NormalTok{y}
\NormalTok{\}}

\CommentTok{# Bad}
\NormalTok{add_two <-}\StringTok{ }\ControlFlowTok{function}\NormalTok{(x, y) \{}
    \KeywordTok{return}\NormalTok{(x }\OperatorTok{+}\StringTok{ }\NormalTok{y)}
\NormalTok{\}}
\end{Highlighting}
\end{Shaded}
\item
  Return statements should always be on their own line because they have
  important effects on the control flow

\begin{Shaded}
\begin{Highlighting}[]
\CommentTok{# Good}
\NormalTok{find_abs <-}\StringTok{ }\ControlFlowTok{function}\NormalTok{(x) \{}
    \ControlFlowTok{if}\NormalTok{ (x }\OperatorTok{>}\StringTok{ }\DecValTok{0}\NormalTok{) \{}
        \KeywordTok{return}\NormalTok{(x)}
\NormalTok{    \}}
\NormalTok{    x }\OperatorTok{*}\StringTok{ }\DecValTok{-1}
\NormalTok{\}}

\CommentTok{# Bad}
\NormalTok{find_abs <-}\StringTok{ }\ControlFlowTok{function}\NormalTok{(x) \{}
    \ControlFlowTok{if}\NormalTok{ (x }\OperatorTok{>}\StringTok{ }\DecValTok{0}\NormalTok{) }\KeywordTok{return}\NormalTok{(x)}
\NormalTok{    x }\OperatorTok{*}\StringTok{ }\DecValTok{-1}
\NormalTok{\}}
\end{Highlighting}
\end{Shaded}
\item
  If your function is called primarily for its side-effects (like printing,
  plotting, or saving to disk), it should return the first argument invisibly.
  This makes it possible to use the function as part of a pipe. print methods
  should usually do this, like this example from httr:

\begin{Shaded}
\begin{Highlighting}[]
\NormalTok{print.url <-}\StringTok{ }\ControlFlowTok{function}\NormalTok{(x, ...) \{}
    \KeywordTok{cat}\NormalTok{(}\StringTok{"Url: "}\NormalTok{, }\KeywordTok{build_url}\NormalTok{(x), }\StringTok{"}\CharTok{\textbackslash{}n}\StringTok{"}\NormalTok{, }\DataTypeTok{sep =} \StringTok{""}\NormalTok{)}
    \KeywordTok{invisible}\NormalTok{(x)}
\NormalTok{\}}
\end{Highlighting}
\end{Shaded}
\end{itemize}

\hypertarget{comments-1}{%
\subsection{Comments}\label{comments-1}}

\begin{itemize}
\item
  In code, use comments to explain the ``why'' not the ``what'' or ``how''. Each
  line of a comment should begin with the comment symbol and a single space:
  \texttt{\#}.

\begin{Shaded}
\begin{Highlighting}[]
\CommentTok{# Good}

\CommentTok{# Objects like data frames are treated as leaves}
\NormalTok{x <-}\StringTok{ }\KeywordTok{map_if}\NormalTok{(x, is_bare_list, recurse)}


\CommentTok{# Bad}

\CommentTok{# Recurse only with bare lists}
\NormalTok{x <-}\StringTok{ }\KeywordTok{map_if}\NormalTok{(x, is_bare_list, recurse)}
\end{Highlighting}
\end{Shaded}
\item
  Comments should be in sentence case, and only end with a full stop if they
  contain at least two sentences:

\begin{Shaded}
\begin{Highlighting}[]
\CommentTok{# Good}

\CommentTok{# Objects like data frames are treated as leaves}
\NormalTok{x <-}\StringTok{ }\KeywordTok{map_if}\NormalTok{(x, is_bare_list, recurse)}

\CommentTok{# Do not use `is.list()`. Objects like data frames must be treated}
\CommentTok{# as leaves.}
\NormalTok{x <-}\StringTok{ }\KeywordTok{map_if}\NormalTok{(x, is_bare_list, recurse)}


\CommentTok{# Bad}

\CommentTok{# objects like data frames are treated as leaves}
\NormalTok{x <-}\StringTok{ }\KeywordTok{map_if}\NormalTok{(x, is_bare_list, recurse)}

\CommentTok{# Objects like data frames are treated as leaves.}
\NormalTok{x <-}\StringTok{ }\KeywordTok{map_if}\NormalTok{(x, is_bare_list, recurse)}
\end{Highlighting}
\end{Shaded}
\end{itemize}

\hypertarget{control-flow}{%
\section{Control flow}\label{control-flow}}

\hypertarget{code-blocks}{%
\subsection{Code blocks}\label{code-blocks}}

\begin{itemize}
\item
  \texttt{\{} should be the last character on the line. Related code (e.g., an \texttt{if}
  clause, a function declaration, a trailing comma, \ldots) must be on the same line
  as the opening brace.
\item
  The contents should be indented by four spaces.
\item
  \texttt{\}} should be the first character on the line.

\begin{Shaded}
\begin{Highlighting}[]
\CommentTok{# good}
\ControlFlowTok{if}\NormalTok{ (y }\OperatorTok{<}\StringTok{ }\DecValTok{0} \OperatorTok{&&}\StringTok{ }\NormalTok{debug) \{}
    \KeywordTok{message}\NormalTok{(}\StringTok{"y is negative"}\NormalTok{)}
\NormalTok{\}}

\ControlFlowTok{if}\NormalTok{ (y }\OperatorTok{==}\StringTok{ }\DecValTok{0}\NormalTok{) \{}
    \ControlFlowTok{if}\NormalTok{ (x }\OperatorTok{>}\StringTok{ }\DecValTok{0}\NormalTok{) \{}
        \KeywordTok{log}\NormalTok{(x)}
\NormalTok{    \} }\ControlFlowTok{else}\NormalTok{ \{}
        \KeywordTok{message}\NormalTok{(}\StringTok{"x is negative or zero"}\NormalTok{)}
\NormalTok{    \}}
\NormalTok{\} }\ControlFlowTok{else}\NormalTok{ \{}
\NormalTok{    y}\OperatorTok{^}\NormalTok{x}
\NormalTok{\}}

\CommentTok{# bad}
\ControlFlowTok{if}\NormalTok{ (y }\OperatorTok{==}\StringTok{ }\DecValTok{0}\NormalTok{)}
\NormalTok{\{}
  \ControlFlowTok{if}\NormalTok{ (x }\OperatorTok{>}\StringTok{ }\DecValTok{0}\NormalTok{) \{}
    \KeywordTok{log}\NormalTok{(x)}
\NormalTok{  \} }\ControlFlowTok{else}\NormalTok{ \{}
\KeywordTok{message}\NormalTok{(}\StringTok{"x is negative or zero"}\NormalTok{)}
\NormalTok{  \}}
\NormalTok{\} }\ControlFlowTok{else}\NormalTok{ \{ y }\OperatorTok{^}\StringTok{ }\NormalTok{x \}}
\end{Highlighting}
\end{Shaded}
\end{itemize}

\hypertarget{if-statements}{%
\subsection{If Statements}\label{if-statements}}

\begin{itemize}
\tightlist
\item
  If used, \texttt{else} should be on the same line as \texttt{\}}.
\item
  \texttt{\&} and \texttt{\textbar{}} should never be used inside of an \texttt{if} clause because they can
  return vectors. Always use \texttt{\&\&} and \texttt{\textbar{}\textbar{}} instead.
\end{itemize}

\hypertarget{inline-statement}{%
\subsection{Inline statement}\label{inline-statement}}

\begin{itemize}
\item
  You can write a simple \texttt{if} block on ane single line

\begin{Shaded}
\begin{Highlighting}[]
\NormalTok{message <-}\StringTok{ }\ControlFlowTok{if}\NormalTok{ (x }\OperatorTok{>}\StringTok{ }\DecValTok{10}\NormalTok{) }\StringTok{"big"} \ControlFlowTok{else} \StringTok{"small"}
\end{Highlighting}
\end{Shaded}
\item
  Function calls that affect control flow (like \texttt{return()}, \texttt{stop()} or
  \texttt{continue}) should always go in their own \texttt{\{\}} block:

\begin{Shaded}
\begin{Highlighting}[]
\CommentTok{# Good}
\ControlFlowTok{if}\NormalTok{ (y }\OperatorTok{<}\StringTok{ }\DecValTok{0}\NormalTok{) \{}
    \KeywordTok{stop}\NormalTok{(}\StringTok{"Y is negative"}\NormalTok{)}
\NormalTok{\}}

\NormalTok{find_abs <-}\StringTok{ }\ControlFlowTok{function}\NormalTok{(x) \{}
    \ControlFlowTok{if}\NormalTok{ (x }\OperatorTok{>}\StringTok{ }\DecValTok{0}\NormalTok{) \{}
        \KeywordTok{return}\NormalTok{(x)}
\NormalTok{    \}}
\NormalTok{    x }\OperatorTok{*}\StringTok{ }\DecValTok{-1}
\NormalTok{\}}

\CommentTok{# Bad}
\ControlFlowTok{if}\NormalTok{ (y }\OperatorTok{<}\StringTok{ }\DecValTok{0}\NormalTok{) }\KeywordTok{stop}\NormalTok{(}\StringTok{"Y is negative"}\NormalTok{)}

\ControlFlowTok{if}\NormalTok{ (y }\OperatorTok{<}\StringTok{ }\DecValTok{0}\NormalTok{)}
    \KeywordTok{stop}\NormalTok{(}\StringTok{"Y is negative"}\NormalTok{)}
\end{Highlighting}
\end{Shaded}
\end{itemize}

\hypertarget{implicit-type-coercion}{%
\subsection{Implicit type coercion}\label{implicit-type-coercion}}

Avoid implicit type coercion (e.g.~from numeric to logical) in \texttt{if} statements:

\begin{Shaded}
\begin{Highlighting}[]
\CommentTok{# Good}
\ControlFlowTok{if}\NormalTok{ (}\KeywordTok{length}\NormalTok{(x) }\OperatorTok{>}\StringTok{ }\DecValTok{0}\NormalTok{) \{}
    \CommentTok{# do something}
\NormalTok{\}}

\CommentTok{# Bad}
\ControlFlowTok{if}\NormalTok{ (}\KeywordTok{length}\NormalTok{(x)) \{}
    \CommentTok{# do something}
\NormalTok{\}}
\end{Highlighting}
\end{Shaded}

\hypertarget{switch-statements}{%
\subsection{Switch statements}\label{switch-statements}}

\begin{itemize}
\item
  Avoid position-based \texttt{switch()} statements (i.e.~prefer names).
\item
  Each element should go on its own line.
\item
  Elements that fall through to the following element should have a space after
  \texttt{=}.
\item
  Provide a fall-through error, unless you have previously validated the input.

\begin{Shaded}
\begin{Highlighting}[]
\CommentTok{# Good }
\ControlFlowTok{switch}\NormalTok{(x, }
    \DataTypeTok{a =}\NormalTok{ ,}
    \DataTypeTok{b =} \DecValTok{1}\NormalTok{, }
    \DataTypeTok{c =} \DecValTok{2}\NormalTok{,}
    \KeywordTok{stop}\NormalTok{(}\StringTok{"Unknown `x`"}\NormalTok{, }\DataTypeTok{call. =} \OtherTok{FALSE}\NormalTok{)}
\NormalTok{)}

\CommentTok{# Bad}
\ControlFlowTok{switch}\NormalTok{(x, }\DataTypeTok{a =}\NormalTok{ , }\DataTypeTok{b =} \DecValTok{1}\NormalTok{, }\DataTypeTok{c =} \DecValTok{2}\NormalTok{)}
\ControlFlowTok{switch}\NormalTok{(x, }\DataTypeTok{a =}\NormalTok{, }\DataTypeTok{b =} \DecValTok{1}\NormalTok{, }\DataTypeTok{c =} \DecValTok{2}\NormalTok{)}
\ControlFlowTok{switch}\NormalTok{(y, }\DecValTok{1}\NormalTok{, }\DecValTok{2}\NormalTok{, }\DecValTok{3}\NormalTok{)}
\end{Highlighting}
\end{Shaded}
\end{itemize}

\hypertarget{long-lines-1}{%
\section{Long lines}\label{long-lines-1}}

\begin{itemize}
\item
  Strive to limit your code to 80 characters per line.
\item
  If a function call is too long to fit on a single line, use one line each for
  the function name, each argument, and the closing \texttt{)}. This makes the code
  easier to read and to change later.

\begin{Shaded}
\begin{Highlighting}[]
\CommentTok{# Good}
\KeywordTok{do_something_very_complicated}\NormalTok{(}
    \DataTypeTok{something =} \StringTok{"that"}\NormalTok{,}
    \DataTypeTok{requires =}\NormalTok{ many,}
    \DataTypeTok{arguments =} \StringTok{"some of which may be long"}
\NormalTok{)}

\CommentTok{# Bad}
\KeywordTok{do_something_very_complicated}\NormalTok{(}\StringTok{"that"}\NormalTok{, requires, many, arguments,}
                              \StringTok{"some of which may be long"}
\NormalTok{                             )}
\end{Highlighting}
\end{Shaded}
\item
  Short unnamed arguments can also go on the same line as the function name,
  even if the whole function call spans multiple lines.

\begin{Shaded}
\begin{Highlighting}[]
\KeywordTok{map}\NormalTok{(x, f,}
    \DataTypeTok{extra_argument_a =} \DecValTok{10}\NormalTok{,}
    \DataTypeTok{extra_argument_b =} \KeywordTok{c}\NormalTok{(}\DecValTok{1}\NormalTok{, }\DecValTok{43}\NormalTok{, }\DecValTok{390}\NormalTok{, }\DecValTok{210209}\NormalTok{)}
\NormalTok{)}
\end{Highlighting}
\end{Shaded}
\item
  You may also place several arguments on the same line if they are closely
  related to each other.

\begin{Shaded}
\begin{Highlighting}[]
\CommentTok{# Good}
\KeywordTok{paste0}\NormalTok{(}
    \StringTok{"Requirement: "}\NormalTok{, requires, }\StringTok{"}\CharTok{\textbackslash{}n}\StringTok{"}\NormalTok{,}
    \StringTok{"Result: "}\NormalTok{, result, }\StringTok{"}\CharTok{\textbackslash{}n}\StringTok{"}
\NormalTok{)}

\CommentTok{# Bad}
\KeywordTok{paste0}\NormalTok{(}
    \StringTok{"Requirement: "}\NormalTok{, requires,}
    \StringTok{"}\CharTok{\textbackslash{}n}\StringTok{"}\NormalTok{, }\StringTok{"Result: "}\NormalTok{,}
\NormalTok{    result, }\StringTok{"}\CharTok{\textbackslash{}n}\StringTok{"}\NormalTok{)}
\end{Highlighting}
\end{Shaded}
\end{itemize}

\hypertarget{semicolons}{%
\section{Semicolons}\label{semicolons}}

Don't put \texttt{;} at the end of a line, and don't use \texttt{;} to put multiple commands
on one line.

\hypertarget{assignment-1}{%
\section{Assignment}\label{assignment-1}}

Use \texttt{\textless{}-}, not \texttt{=}, for assignment.

\hypertarget{character-vectors}{%
\section{Character vectors}\label{character-vectors}}

Use \texttt{"}, not \texttt{\textquotesingle{}}, for quoting text. The only exception is when the text already
contains double quotes and no single quotes.

\begin{verbatim}
# Good
"Text"
'Text with "quotes"'
'<a href="http://style.tidyverse.org">A link</a>'

# Bad
'Text'
'Text with "double" and \'single\' quotes'
\end{verbatim}

\hypertarget{logical-vectors}{%
\section{Logical vectors}\label{logical-vectors}}

Use \texttt{TRUE} and \texttt{FALSE} rather than \texttt{T} and \texttt{F}.

\hypertarget{comments-2}{%
\section{Comments}\label{comments-2}}

Each line of a comment should begin with the comment symbol \texttt{\#} and a single space.

In data analysis code, use comments to record important findings and analysis
decisions. If you need comments to explain what your code is doing, consider
rewriting your code to be clearer. If you discover that you have more comments
than code, consider switching to \href{https://rmarkdown.rstudio.com/}{R Markdown}.

\hypertarget{resources-1}{%
\section{Resources}\label{resources-1}}

\begin{itemize}
\tightlist
\item
  \href{https://github.com/r-lib/styler}{styler}: formatting your code according to
  the tidyverse style guide (or your custom style guide) so you can direct your
  attention to the content of your code. It helps to keep the coding style
  consistent across projects and facilitate collaboration.
\item
  \href{https://github.com/r-lib/lintr}{lintr}: offering static code analysis for R.
  It checks adherence to a given style, syntax errors and possible semantic
  issues.
\end{itemize}

\end{document}
